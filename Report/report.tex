\documentclass[12pt,a4paper]{article}
\usepackage[utf8]{inputenc}
\usepackage[T1]{fontenc}
\usepackage{geometry}
\usepackage{graphicx}
\usepackage{amsmath}
\usepackage{amsfonts}
\usepackage{amssymb}
\usepackage{hyperref}
\usepackage{listings}
\usepackage{xcolor}
\usepackage{booktabs}
\usepackage{array}
\usepackage{float}
\usepackage{enumitem}
\usepackage{fancyhdr}
\usepackage{titlesec}
\usepackage{subcaption}
\usepackage{tcolorbox}

% Page setup
\geometry{margin=2.5cm}
\pagestyle{fancy}
\fancyhf{}
\fancyhead[L]{Multimedia Search Engine}
\fancyhead[R]{UMONS 2024-2025}
\fancyfoot[C]{\thepage}

% Colors for code
\definecolor{codegreen}{rgb}{0,0.6,0}
\definecolor{codegray}{rgb}{0.5,0.5,0.5}
\definecolor{codepurple}{rgb}{0.58,0,0.82}
\definecolor{backcolour}{rgb}{0.95,0.95,0.92}
\definecolor{securityblue}{rgb}{0.1,0.3,0.7}

% Code listing style
\lstdefinestyle{mystyle}{
    backgroundcolor=\color{backcolour},   
    commentstyle=\color{codegreen},
    keywordstyle=\color{magenta},
    numberstyle=\tiny\color{codegray},
    stringstyle=\color{codepurple},
    basicstyle=\ttfamily\footnotesize,
    breakatwhitespace=false,         
    breaklines=true,                 
    captionpos=b,                    
    keepspaces=true,                 
    numbers=left,                    
    numbersep=5pt,                  
    showspaces=false,                
    showstringspaces=false,
    showtabs=false,                  
    tabsize=2
}
\lstset{style=mystyle}

% Title formatting
\titleformat{\section}{\Large\bfseries\color{blue!70!black}}{\thesection}{1em}{}
\titleformat{\subsection}{\large\bfseries\color{blue!50!black}}{\thesubsection}{1em}{}

% Implementation status boxes
\newtcolorbox{implementedbox}{
    colback=green!5!white,
    colframe=green!50!black,
    title=\textbf{✅ Implémenté},
    fonttitle=\bfseries,
    rounded corners
}

\newtcolorbox{partialbox}{
    colback=orange!5!white,
    colframe=orange!50!black,
    title=\textbf{⚠️ Partiellement Implémenté},
    fonttitle=\bfseries,
    rounded corners
}

\newtcolorbox{futurebox}{
    colback=blue!5!white,
    colframe=blue!50!black,
    title=\textbf{�� Amélioration Future},
    fonttitle=\bfseries,
    rounded corners
}

\begin{document}

% Title page
\begin{titlepage}
    \centering
    \vspace*{2cm}
    
    {\Huge\bfseries Multimedia Search Engine}\\[0.5cm]
    {\LARGE Content-Based Image Retrieval using Deep Learning}\\[1.5cm]
    
    {\Large\bfseries Projet de l'AA: Machine \& Deep Learning for Multimedia Retrieval}\\[1cm]
    {\large 2024-2025}\\[2cm]
    
    {\Large\textbf{Auteurs:}}\\[0.5cm]
    {\large Abdelhadi Agourzam}\\
    {\large Mohammed El-Ismayily}\\[2cm]
    
    {\Large\textbf{Superviseurs:}}\\[0.5cm]
    {\large Prof. Sidi Ahmed Mahmoudi}\\
    {\large Dr. Aurélie Cools}\\
    {\large Maxime Gloesener}\\[2cm]
    
    \includegraphics[width=0.3\textwidth]{umons_logo.png}\\[1cm]
    
    {\large Université de Mons (UMONS)}\\
    {\large Faculté Polytechnique}\\
    {\large Cloud \& Edge Computing}
    
    \vfill
    {\large \today}
\end{titlepage}

\tableofcontents
\newpage

\section{Introduction}

Ce rapport présente un moteur de recherche multimédia basé sur l'apprentissage profond, développé dans le cadre du cours "Machine \& Deep Learning for Multimedia Retrieval" à l'Université de Mons (UMONS). Le projet constitue un système de recherche d'images par contenu utilisant plusieurs modèles de réseaux de neurones convolutionnels, avec un déploiement sur infrastructure cloud utilisant Docker.

Cette solution représente un système fonctionnel intégrant l'intelligence artificielle, une interface utilisateur moderne et un déploiement cloud, enrichi de fonctionnalités de sécurité de base illustrant les bonnes pratiques de développement.

\subsection{Objectifs du Projet}

L'objectif principal était de créer un moteur de recherche d'images performant exploitant plusieurs descripteurs deep learning avec différentes métriques de similarité. Le système devait proposer une interface web professionnelle et être déployé sur infrastructure cloud avec Docker, tout en permettant l'évaluation des performances via des métriques de précision et rappel.

\section{Architecture du Système}

\subsection{Modèles d'Intelligence Artificielle Intégrés}

\begin{implementedbox}
Le système exploite trois modèles d'intelligence artificielle complémentaires :

\textbf{VGG16} excelle dans la reconnaissance de textures fines grâce à son architecture séquentielle de 16 couches. Ce modèle capture efficacement les détails visuels et les motifs répétitifs des images.

\textbf{ResNet50} apporte sa capacité à analyser des relations spatiales complexes grâce à ses connexions résiduelles qui permettent un apprentissage profond sans dégradation du gradient.

\textbf{MobileNet} offre une solution optimisée en termes d'efficacité computationnelle tout en maintenant de bonnes performances d'extraction de caractéristiques.
\end{implementedbox}

Cette approche multi-modèles permet d'exploiter les forces spécifiques de chaque architecture pour obtenir des résultats de recherche plus robustes et précis.

\subsection{Métriques de Similarité Avancées}

\begin{implementedbox}
Le système implémente quatre métriques de similarité qui offrent différentes perspectives sur la comparaison d'images :

La \textbf{distance euclidienne} fournit une mesure directe de proximité géométrique entre les vecteurs de caractéristiques. La \textbf{similarité cosinus} se concentre sur l'orientation des vecteurs, rendant la mesure invariante aux variations d'intensité. La \textbf{distance chi-carré} apporte une perspective statistique particulièrement efficace pour les histogrammes. La \textbf{distance de Bhattacharyya} offre une mesure sophistiquée originellement conçue pour comparer des distributions de probabilité.
\end{implementedbox}

\subsection{Fonctionnalités de Recherche}

\begin{implementedbox}
Le système propose deux modes de recherche principaux :

\textbf{Recherche individuelle} permet d'explorer les capacités spécifiques de chaque modèle avec génération automatique de courbes Précision-Rappel pour évaluer la qualité des résultats.

\textbf{Recherche combinée} constitue l'innovation principale, fusionnant intelligemment les résultats de plusieurs modèles selon trois méthodes : moyenne simple, combinaison pondérée avec contrôles personnalisables, et fusion de rangs utilisant l'algorithme Reciprocal Rank Fusion.
\end{implementedbox}

\section{Architecture Technique et Infrastructure}

\subsection{Architecture Microservices}

\begin{figure}[h]
  \centering
  \includegraphics[width=1\textwidth]{1.png}
  \caption{Architecture microservices du moteur de recherche multimédia}
  \label{fig:architecture}
\end{figure}

\begin{implementedbox}
L'architecture repose sur une structure microservices composée de trois services principaux orchestrés par Docker Compose :

\begin{itemize}
    \item \textbf{Application Flask} : Gère la logique métier et l'API REST
    \item \textbf{Redis} : Assure l'authentification et la mise en cache
    \item \textbf{Nginx} : Fonctionne comme proxy inverse avec optimisations de performance
\end{itemize}

Cette architecture garantit la modularité, la maintenabilité et facilite le déploiement du système.
\end{implementedbox}

\subsection{Sécurité et Authentification}

\begin{implementedbox}
\textbf{Système de Hachage Sécurisé}

Le système intègre un mécanisme de hachage cryptographique utilisant SHA-256 avec salage aléatoire pour la protection des mots de passe utilisateur :

\begin{lstlisting}[language=Python, caption=Implémentation du hachage sécurisé]
def add_user_to_redis(username, password):
    # Génération d'un sel aléatoire de 8 bytes
    salt = secrets.token_hex(8)
    
    # Hachage SHA-256 avec sel
    password_with_salt = password + salt
    hashed_password = hashlib.sha256(password_with_salt.encode()).hexdigest()
    
    # Stockage au format hash:salt
    stored_value = f"{hashed_password}:{salt}"
    redis_client.hset("users", username, stored_value)
\end{lstlisting}
\end{implementedbox}
\begin{implementedbox}
\textbf{Caractéristiques de Sécurité :}
\begin{itemize}
    \item \textbf{Hachage SHA-256} : Fonction cryptographique sécurisée et irréversible
    \item \textbf{Salage aléatoire} : Sel unique de 16 caractères hexadécimaux par mot de passe
    \item \textbf{Migration automatique} : Conversion transparente des mots de passe existants
    \item \textbf{Stockage sécurisé} : Format hash:salt dans Redis, empêchant la récupération du mot de passe original
\end{itemize}
\end{implementedbox}

\begin{partialbox}
\textbf{Gestion des Sessions (Basique)}

Le système utilise des sessions Flask avec stockage Redis :
\begin{itemize}
    \item Sessions avec expiration automatique (30 minutes)
    \item Persistance lors des redémarrages de conteneurs
    \item Déconnexion basique avec nettoyage de session
\end{itemize}

\textbf{Limitations identifiées :}
\begin{itemize}
    \item Protection CSRF avec tokens non implémentée
    \item Logging des tentatives d'authentification minimal
    \item Invalidation sécurisée avancée non incluse
\end{itemize}
\end{partialbox}

Le système propose quatre profils utilisateur : administrateur, chercheur, étudiant et utilisateur de démonstration, adaptés aux besoins d'un environnement universitaire. De nouveaux utilisateurs peuvent être ajoutés via l'API dédiée.

\begin{figure}[H]
  \centering
  \includegraphics[width=0.8\textwidth]{login.png}
  \caption{Interface de connexion}
  \label{fig:login_interface}
\end{figure}

\subsection{Interface Utilisateur}

\begin{implementedbox}
L'interface principale présente un tableau de bord complet avec :
\begin{itemize}
    \item Statistiques en temps réel
    \item Contrôles de recherche sophistiqués
    \item Affichage des résultats avec visualisations interactives
    \item Design responsive pour tous les appareils
\end{itemize}

Les contrôles incluent la sélection d'images requête, la configuration des paramètres de recherche, la sélection multiple des modèles AI et les options avancées de combinaison avec pondération dynamique.
\end{implementedbox}

\begin{figure}[H]
  \centering
  \includegraphics[width=0.8\textwidth]{main.png}
  \caption{Interface principale interactive}
  \label{fig:main_interface}
\end{figure}

\section*{Fonctionnalités de l'Interface}

\textbf{Tableau de Bord Statistiques}
\begin{itemize}[leftmargin=2em]
  \item \textbf{Total Images} : Nombre d'images indexées
  \item \textbf{AI Models} : Modèles disponibles (3)
  \item \textbf{Similarity Metrics} : Métriques supportées (4)
  \item \textbf{Searches Performed} : Compteur de recherches
\end{itemize}

\textbf{Contrôles de Recherche}
\begin{itemize}[leftmargin=2em]
  \item \textbf{Sélection d'image requête} : Liste déroulante + bouton \emph{Random}
  \item \textbf{Métrique de similarité} : 4 options (Euclidienne, Cosinus, Chi-carré, Bhattacharyya)
  \item \textbf{Nombre de résultats} : Top 10 / 20 / 50
  \item \textbf{Sélection de modèles} : VGG16, ResNet50, MobileNet (multi-sélection)
\end{itemize}

\textbf{Options Avancées (si 2+ modèles)}
\begin{itemize}[leftmargin=2em]
  \item \textbf{Méthodes de combinaison} : Average, Weighted, Rank Fusion
  \item \textbf{Contrôles de pondération} : Sliders dynamiques pour chaque modèle
  \item \textbf{Configuration intelligente} : Interface adaptative
\end{itemize}

\subsection{Affichage des Résultats}

\begin{implementedbox}
Les résultats sont organisés en onglets séparés pour chaque modèle, permettant une exploration comparative détaillée. Chaque recherche génère automatiquement des courbes Précision-Rappel interactives de haute qualité avec des métriques de performance complètes.

Pour les recherches combinées, un onglet spécialisé présente les résultats de fusion avec analyse comparative des différentes stratégies de combinaison.
\end{implementedbox}

\begin{figure}[H]
  \centering
  \includegraphics[width=0.8\textwidth]{choose_pic.png}
  \caption{Interface de sélection et lancement des modèles}
  \label{fig:model_selection}
\end{figure}

\begin{figure}[H]
  \centering
  \includegraphics[width=0.8\textwidth]{courbes.png}
  \caption{Courbes Précision-Rappel automatiquement générées}
  \label{fig:precision_recall}
\end{figure}

\section{Infrastructure et Déploiement Cloud}

\subsection{API REST}

\begin{implementedbox}
Le système expose une API REST suivant les standards modernes avec :

\textbf{Routes d'Authentification}
\begin{itemize}[leftmargin=2em]
  \item \texttt{/login} (GET/POST) : Connexion avec vérification de hachage
  \item \texttt{/logout} (GET) : Déconnexion avec nettoyage de session
\end{itemize}

\textbf{Route Principale}
\begin{itemize}[leftmargin=2em]
  \item \texttt{/} (GET) : Interface principale (protégée par authentification)
\end{itemize}

\textbf{API de Recherche}
\begin{itemize}[leftmargin=2em]
  \item \texttt{/api/search} (POST) : Recherche individuelle par modèle
  \item \texttt{/api/search\_combined} (POST) : Recherche combinée multi-modèles
\end{itemize}

\textbf{API de Support}
\begin{itemize}[leftmargin=2em]
  \item \texttt{/api/available\_images} (GET) : Liste des images disponibles
  \item \texttt{/api/stats} (GET) : Statistiques système en temps réel
  \item \texttt{/api/add\_user} (POST) : Ajout d'utilisateurs avec hachage automatique
  \item \texttt{/health} (GET) : Endpoint de santé pour monitoring Docker
\end{itemize}
\end{implementedbox}

\subsection{Configuration Docker avec Volumes Persistants}

\begin{implementedbox}
La configuration Docker repose sur une orchestration de trois services avec persistance des données :

\paragraph{1. Dockerfile}
\begin{itemize}
  \item Image légère \texttt{python:3.11-slim}
  \item Stratégie de couches optimisée avec cache Docker
  \item Installation ciblée des dépendances
  \item Health checks intégrés via endpoint \texttt{/health}
\end{itemize}

\paragraph{2. docker-compose.yml}
Orchestration de trois services avec volumes persistants :
\begin{itemize}
  \item \textbf{Redis} : avec volume \texttt{redis\_data:/data} pour persistance des sessions et hachages
  \item \textbf{Flask app} : avec volume \texttt{models\_cache:/app/models\_cache} pour cache des modèles
  \item \textbf{Nginx} : avec volume \texttt{nginx\_cache:/var/cache/nginx} pour cache web
\end{itemize}

\paragraph{3. Gestion des Volumes}
\begin{verbatim}
volumes:
  redis_data:      # Sessions utilisateur et hachages sécurisés
  models_cache:    # Cache des modèles AI (gain 30-45s au démarrage)
  nginx_cache:     # Cache statique pour performances web
\end{verbatim}

\textbf{Avantages des Volumes :}
\begin{itemize}
  \item Persistance des données entre redémarrages
  \item Démarrage rapide grâce au cache des modèles
  \item Conservation des sessions utilisateur
  \item Performances web optimisées
\end{itemize}
\end{implementedbox}

\begin{partialbox}
\textbf{Sécurité Infrastructure Basique}
\begin{itemize}
  \item \textbf{✅ Implémenté} : Réseau isolé \texttt{multimedia\_network}
  \item \textbf{✅ Implémenté} : Conteneurisation des services
  \item \textbf{✅ Implémenté} : Health checks automatiques
  \item \textbf{❌ Limitation} : Utilisateurs non-privilégiés dans conteneurs
  \item \textbf{❌ Limitation} : Chiffrement inter-services
\end{itemize}
\end{partialbox}

\subsection{Configuration Nginx}

\begin{implementedbox}
La configuration Nginx assure performance et sécurité de base :

\textbf{Optimisations de Performance}
\begin{itemize}[leftmargin=2em]
  \item \texttt{sendfile}, \texttt{tcp\_nopush}, \texttt{tcp\_nodelay} activés
  \item Compression Gzip pour réduire la bande passante (60-80\%)
  \item Cache statique intelligent avec \texttt{expires 1h}
  \item Proxy inverse avec connexions persistantes
\end{itemize}

\textbf{Timeouts Adaptés}
\begin{itemize}[leftmargin=2em]
  \item 10s pour health checks rapides
  \item 300s pour recherches intensives
  \item 60s pour opérations standard
\end{itemize}

\textbf{Sécurité de Base}
\begin{itemize}[leftmargin=2em]
  \item Headers de sécurité HTTP basiques
  \item Protection contre clickjacking et MIME sniffing
  \item Filtrage des requêtes malformées
\end{itemize}
\end{implementedbox}

\subsection{Déploiement sur Infrastructure Cloud}

\begin{implementedbox}
Déploiement réalisé sur infrastructure Proxmox VE :
\begin{itemize}
  \item VM Ubuntu Server 22.04 LTS (4 vCPU, 8GB RAM, 50GB SSD)
  \item Script de déploiement automatisé \texttt{deployment.sh}
  \item Vérification des prérequis et nettoyage intelligent
  \item Surveillance progressive des services avec validation
  \item Firewall UFW configuré pour ports essentiels
\end{itemize}
\end{implementedbox}

\section{Performances et Évaluation}

\subsection{Métriques de Performance}

\begin{implementedbox}
Le système démontre des performances satisfaisantes avec des temps de réponse mesurés :

\begin{table}[H]
\centering
\begin{tabular}{|l|c|c|}
\hline
\textbf{Métrique} & \textbf{Valeur Mesurée} & \textbf{Méthode} \\
\hline
Temps recherche individuelle & 2.3s & performance.now() JavaScript \\
Temps recherche combinée & 5-7s & performance.now() JavaScript \\
Cache hit Redis & <500ms & Mesure backend \\
Consommation mémoire & 2.5-4.2GB & Observation système \\
Hachage mot de passe & <10ms & Timing Python \\
\hline
\end{tabular}
\caption{Métriques de Performance Mesurées}
\end{table}

L'utilisation du cache Redis réduit significativement les temps de réponse pour les requêtes fréquentes. L'efficacité mémoire permet un déploiement sur des infrastructures cloud standard.
\end{implementedbox}

\subsection{Qualité de Recherche}

\begin{implementedbox}
L'évaluation révèle des performances de recherche satisfaisantes :

\begin{table}[H]
\centering
\begin{tabular}{|l|c|c|c|}
\hline
\textbf{Modèle} & \textbf{Précision Moyenne} & \textbf{Rappel Moyen} & \textbf{MAP Score} \\
\hline
VGG16 & 0.72-0.84 & 0.68-0.78 & 0.74 \\
ResNet50 & 0.78-0.89 & 0.74-0.85 & 0.79 \\
MobileNet & 0.75-0.86 & 0.71-0.82 & 0.76 \\
\hline
\textbf{Combiné} & \textbf{0.81-0.91} & \textbf{0.76-0.88} & \textbf{0.83} \\
\hline
\end{tabular}
\caption{Performance de Qualité par Modèle}
\end{table}

L'approche de recherche combinée améliore les performances avec des scores MAP moyens de 0.83 comparés à 0.76 pour les recherches individuelles, validant la stratégie multi-modèles.
\end{implementedbox}

\section{État Actuel et Limitations}

\subsection{Fonctionnalités Implémentées}

\begin{implementedbox}
\textbf{✅ Fonctionnalités Complètement Implémentées}
\begin{itemize}
  \item Moteur de recherche multi-modèles (VGG16, ResNet50, MobileNet)
  \item Quatre métriques de similarité fonctionnelles
  \item Interface web complète et responsive
  \item API REST avec tous les endpoints
  \item Hachage sécurisé des mots de passe (SHA-256 + salt)
  \item Déploiement Docker avec volumes persistants
  \item Courbes Précision-Rappel automatiques
  \item Recherche combinée avec trois méthodes de fusion
  \item Cache Redis pour optimisation des performances
\end{itemize}
\end{implementedbox}

\subsection{Limitations Identifiées}

\begin{partialbox}
\textbf{⚠️ Fonctionnalités Basiques}
\begin{itemize}
  \item Sessions avec timeout basique (protection CSRF limitée)
  \item Logging minimal des actions
  \item Sécurité infrastructure de base
  \item Monitoring simple via health checks
  \item Gestion d'erreurs basique
\end{itemize}
\end{partialbox}


\section{Contributions et Innovation}

\begin{implementedbox}
Le code source complet est disponible publiquement sur GitHub à l'adresse \url{https://github.com/oscarRickovic/Cloud_uni.git}, favorisant la transparence académique et la reproductibilité.

\textbf{Contributions Techniques}
\begin{itemize}
  \item \textbf{Architecture multi-modèles} : Fusion de trois modèles CNN avec stratégies de combinaison
  \item \textbf{Interface adaptative} : Design responsive avec visualisations interactives
  \item \textbf{Persistance optimisée} : Volumes Docker pour performance et continuité
  \item \textbf{Sécurité intégrée} : Hachage cryptographique dès le développement
  \item \textbf{Déploiement reproductible} : Infrastructure as Code avec Docker
\end{itemize}
\end{implementedbox}

\section{Configuration Technique Détaillée}

\subsection{Docker Compose}

\begin{lstlisting}[language=yaml, caption=docker-compose.yml avec volumes persistants]
version: '3.8'
services:
  redis:
    image: redis:7-alpine
    container_name: multimedia_redis
    ports:
      - "6379:6379"
    volumes:
      - redis_data:/data
    networks:
      - multimedia_network
    restart: unless-stopped
    healthcheck:
      test: ["CMD", "redis-cli", "ping"]
      interval: 30s
      timeout: 10s
      retries: 3

  flask_app:
    build: .
    container_name: multimedia_flask
    ports:
      - "5000:5000"
    volumes:
      - ./features:/app/features:ro
      - ./image.orig:/app/image.orig:ro
      - models_cache:/app/models_cache
    environment:
      - REDIS_HOST=redis
      - REDIS_PORT=6379
      - FLASK_ENV=production
    depends_on:
      redis:
        condition: service_healthy
    networks:
      - multimedia_network
    restart: unless-stopped

  nginx:
    image: nginx:alpine
    container_name: multimedia_nginx
    ports:
      - "80:80"
    volumes:
      - ./nginx.conf:/etc/nginx/conf.d/default.conf:ro
      - ./image.orig:/usr/share/nginx/html/images:ro
      - nginx_cache:/var/cache/nginx
    depends_on:
      - flask_app
    networks:
      - multimedia_network
    restart: unless-stopped

volumes:
  redis_data:
  models_cache:
  nginx_cache:

networks:
  multimedia_network:
    driver: bridge
\end{lstlisting}

\subsection{Implémentation Sécurité}

\begin{lstlisting}[language=Python, caption=Fonctions de sécurité]
import hashlib
import secrets
from functools import wraps

def add_user_to_redis(username, password):
    """Ajoute un utilisateur avec mot de passe haché"""
    if redis_client:
        try:
            # Génération d'un sel cryptographiquement sûr
            salt = secrets.token_hex(8)  # 16 caractères hex
            
            # Hachage avec sel
            password_with_salt = password + salt
            hashed_password = hashlib.sha256(
                password_with_salt.encode()
            ).hexdigest()
            
            # Stockage format hash:salt
            stored_value = f"{hashed_password}:{salt}"
            redis_client.hset("users", username, stored_value)
            print(f"✅ User {username} added with hashed password")
            return True
        except Exception as e:
            print(f"❌ Error adding user: {e}")
    return False

def verify_redis_password(username, password):
    """Vérifie le mot de passe contre le hachage stocké"""
    if redis_client:
        try:
            stored_value = redis_client.hget("users", username)
            if stored_value:
                stored_value = stored_value.decode('utf-8') \
                    if isinstance(stored_value, bytes) else stored_value
                
                # Vérification format hash:salt
                if ':' in stored_value:
                    stored_hash, salt = stored_value.split(':')
                    password_with_salt = password + salt
                    password_hash = hashlib.sha256(
                        password_with_salt.encode()
                    ).hexdigest()
                    return password_hash == stored_hash
                else:
                    # Migration automatique plain text -> hash
                    if stored_value == password:
                        print(f"�� Auto-migrating password for {username}")
                        redis_client.hdel("users", username)
                        add_user_to_redis(username, password)
                        return True
        except Exception as e:
            print(f"❌ Password verification error: {e}")
    return False

def login_required(f):
    """Décorateur pour protection des routes"""
    @wraps(f)
    def decorated_function(*args, **kwargs):
        if 'username' not in session:
            return redirect(url_for('login'))
        return f(*args, **kwargs)
    return decorated_function

@app.route('/login', methods=['GET', 'POST'])
def login():
    """Route de connexion avec vérification sécurisée"""
    if request.method == 'POST':
        username = request.form.get('username', '').strip()
        password = request.form.get('password', '')
        
        # Vérification avec hachage sécurisé
        if verify_redis_password(username, password):
            session['username'] = username
            session.permanent = True
            app.permanent_session_lifetime = 1800  # 30 minutes
            print(f"✅ Login successful: {username}")
            return redirect(url_for('index'))
        else:
            print(f"❌ Login failed: {username}")
            return render_template_string(
                get_login_template(), 
                error="Invalid credentials"
            )
    
    # Redirection si déjà connecté
    if 'username' in session:
        return redirect(url_for('index'))
    
    return render_template_string(get_login_template())

@app.route('/logout')
def logout():
    """Déconnexion avec nettoyage de session"""
    username = session.get('username', 'unknown')
    session.clear()
    print(f"�� User {username} logged out")
    return redirect(url_for('login'))
\end{lstlisting}

\subsection{Structure du Projet}

\begin{lstlisting}[caption=Arborescence complète du projet]
multimedia-search-engine/
├── app.py                 # Application Flask principale
├── Dockerfile            # Configuration conteneur
├── docker-compose.yml    # Orchestration services
├── nginx.conf           # Configuration proxy reverse
├── requirements.txt     # Dépendances Python
├── deployment.sh        # Script déploiement automatisé
├── .dockerignore       # Exclusions build Docker
├── README.md           # Documentation projet
├── features/           # Descripteurs extraits des modèles
│   ├── vgg16_features.npy
│   ├── resnet50_features.npy
│   └── mobilenet_features.npy
├── image.orig/         # Dataset d'images
│   ├── image_001.jpg
│   ├── image_002.jpg
│   └── ...
├── models_cache/       # Cache modèles (volume Docker)
└── static/            # Ressources web statiques
    ├── css/
    ├── js/
    └── images/
\end{lstlisting}

\subsection{Exemple de Logs Système}

\begin{lstlisting}[caption=Logs de fonctionnement du système]
[INFO] Starting Multimedia Search Engine...
[INFO] Loading AI models:
[INFO] - VGG16 loaded with 1024 features
[INFO] - ResNet50 loaded with 2048 features  
[INFO] - MobileNet loaded with 1280 features
[INFO] Models cache saved to /app/models_cache/
[INFO] Redis connection established
[INFO] Default users initialized with hashed passwords
[SUCCESS] Application ready on port 5000

[LOGIN] ✅ Login successful: admin
[SEARCH] Individual search VGG16 completed in 2.34s
[SEARCH] Combined search (3 models) completed in 6.12s
[CACHE] Redis hit rate: 78%
[PERFORMANCE] Memory usage: 3.2 GB
[SECURITY] �� Auto-migrating password for user: researcher
[LOGOUT] �� User admin logged out
[HEALTH] All services healthy

[DOCKER] Container multimedia_flask: running
[DOCKER] Container multimedia_redis: running  
[DOCKER] Container multimedia_nginx: running
[VOLUMES] redis_data: 45MB, models_cache: 2.1GB, nginx_cache: 128MB
\end{lstlisting}

\section{Conclusion}
Ce projet représente une réalisation technique qui répond aux exigences académiques tout en intégrant des bonnes pratiques de développement moderne. L'intégration de multiples modèles d'IA avec des méthodes de fusion sophistiquées crée un système fonctionnel et robuste.
\textbf{Points forts de l'implémentation :}
\begin{itemize}
  \item Architecture microservices bien structurée
  \item Sécurité de base avec hachage cryptographique des mots de passe
  \item Interface utilisateur professionnelle et intuitive
  \item Persistance des données avec volumes Docker
  \item Déploiement automatisé et reproductible
  \item Performance satisfaisante pour un environnement académique
\end{itemize}

\textbf{Apprentissages et perspectives :}

Cette réalisation a permis de comprendre les défis du développement d'applications multimédia complètes, depuis l'implémentation des algorithmes d'IA jusqu'au déploiement cloud sécurisé. Les résultats obtenus valident l'approche multi-modèles et ouvrent des perspectives pour de futurs développements.

Cette expérience démontre qu'un projet académique peut intégrer des standards professionnels tout en constituant un excellent support d'apprentissage pour les technologies modernes d'IA et de cloud computing.


\section*{Remerciements}

Nous tenons à remercier l'équipe pédagogique du cours "Cloud and Edge computing" pour leur encadrement, ainsi que l'Université de Mons pour avoir fourni l'infrastructure nécessaire à la réalisation de ce projet.

\end{document}
